\documentclass{article}

\usepackage{amsmath, amssymb}
\usepackage{multicol}
\usepackage{geometry}[margin=1in]

\begin{document}
We are trying to get \(c[n, w]\) based on the recursive relations below.

\begin{description}
    \item[Input] \(n\) items where \(i\)-th item has value \(v_i\) and weighs \(w_i\)
    \item[Output] the maximum value of items that can be carried in a knapsack of capacity \(w\)
\end{description}

\begin{equation*}
    c[i, w] = \begin{cases}
        0                                      & \text{if }i = 0 \text{ or } w = 0        \\
        c[i - 1, w]                            & \text{if }w_i > w                        \\
        \max\{v_i + c[i-1, w-w_i], c[i-1, w]\} & \text{if } i > 0 \text{ and } w \geq w_i
    \end{cases}
\end{equation*}

\begin{multicols}{2}
    \begin{tabular}{c|cccccc}
          & 0 & 1 & 2  & 3  & 4  & 5  \\
        \hline
        0 & 0 & 0 & 0  & 0  & 0  & 0  \\
        1 & 0 & 6 & 6  & 6  & 6  & 6  \\
        2 & 0 & 6 & 10 & 16 & 16 & 16 \\
        3 & 0 & 6 & 10 & 16 & 18 & 22
    \end{tabular}
    \columnbreak
    \begin{align*}
        c[2,1]  & = c[1,1] \text{because } w_2 > w                      \\
        c[2,2]  & = \max(v_2 + c[1,0], c[1,2])                   & = 10 \\
        c[2, 3] & = \max(v_2 + c[1,1], c[1,3]) = \max(10 + 6, 6) & = 16 \\
        c[2,4]  & = \max(v_2 + c[1,2], c[1,4]) = \max(10+6, 6)   & = 16 \\
        c[3,3]  & = \max(v_3 + c[2,0], c[2,3]) = \max(12, 16)    & = 16
    \end{align*}
\end{multicols}

\subsection*{Fractional Kanpsack Problem}
\begin{description}
    \item[Input]  \(n\) items where \(i\)-th item has value \(v_i\) and weighs \(w_i\) (\(v_i\) and \(w_i\) are positive integers)
    \item[Output] The maximum value of items that can be carried in a knapsack of capacity \(w\)
\end{description}

Greedy Algorithm: at each iteration, choose the item with the highest \(\frac{v_i}{w_i}\) and continue when \(W - w_i > 0\)

\subsubsection*{Subproblems}
\begin{itemize}
    \item \(F-KP(i, w)\) fractional kanpsack problem within \(w\) capacity for the first \(i\) items
    \item Goal: \(F-KP(n, W)\)
\end{itemize}

\subsection*{Optimal Substructure}
Suppose OPT is an optimal solution to \(F-KP(i, w)\), there are 2 cases:
\begin{enumerate}
    \item Full/partial item \(i\) in OPT \\
          Remove \(w'\) of item \(i\) from OPT is an optimal solution of \(F-KP(i -1, w-w')\)
    \item Item \(i\) is not in OPT \\
          OPT is an optimal solution of \(F-KP(i-1, w)\)
\end{enumerate}

\subsection*{Proof}
[Let \(j\) be the item with the maximum \(v_i / w_i\).
    Then there exists an optimal solution in which you take as much of item \(j\) as possible.]

Suppose there exists an optimal solution in which you didn't take as much of item \(j\) as possible.

\begin{enumerate}
    \item If the knapsack is not full, add some more of item \(j\), and you have a higher value solution
    \item There must exist some item \(k \neq j\) with \(\frac{v_k}{w_k} < \frac{v_j}{w_j}\) that is in the knapsack.
          %   Replace \(k\) with \(j\) and you have a higher value solution
    \item We can therefore take a piece of \(k\) with \(\epsilon\) weight, out
          of the knapsack, and put a piece of \(j\) with \(\epsilon\) weight in and
          increase the knapsack value.
\end{enumerate}

\subsection*{Graph Terminology}
\begin{align*}
    \text{Graph } G = (V, E)                                     \\
    V & = \text{ set of verticies (or node)}                     \\
    E & = \text{ set of edges (or links)} \subseteq (V \times V)
\end{align*}
\begin{align*}
    V & = \{1, 2, 3, 4, 5\}                                  \\
    E & = \{(1, 2), (1, 3), (1, 4), (2, 4), (2, 5), (4, 5)\}
\end{align*}

Undirected vs Directed
\begin{description}
    \item[Undirected]  edge \((u, v) = (v, u); \quad \forall v, (v, v) \notin E\)  (No self loops)
    \item[Directed] edge \((u, v)\) goes from vertex \(u\) to \(v\)
\end{description}

Unweighted vs Weighted \\
Weighted: Graph associates weights with edges

\subsubsection*{Graph Degrees}
The degree of a vertex \(u\) is th enumber of edges incident to \(u\)
\begin{description}
    \item[Event verticies]  verticies with even degrees
    \item[Odd verticies] verticies with odd degrees
\end{description}

In a directed graph
\begin{description}
    \item[In-degree of \(u\)]  the number of edges entering \(u\)
    \item[Out-degree of \(u\)] the number of edges leaving \(u\)
\end{description}

Handshaking Lemma:

If \(G = (V, E)\) is an undirected grpah, then

\begin{equation*}
    \sum_{v \in V} degree(v) = 2|E|
\end{equation*}

Every undirected graph has an even number of odd verticies.

\subsection*{Representation}

\begin{description}
    \item[Adjacency Matrix]  \(A = (a_{ij})\) where \(a_{ij} = 1\) if \((i, j) \in E\)
    \item[Adjacency List]  \(Adj[u]\) is the list of verticies adjacent to \(u\)
\end{description}
\end{document}