\documentclass{article}

\usepackage{amsmath, amssymb}

\begin{document}
\section*{Review}
\subsection*{Dynamic Programming}
\begin{itemize}
    \item Divide into sub-problems
    \item Solve sub-problems first, then combine to solve larger problem
    \item The sub-problems are overlapping
          \begin{itemize}
              \item Divide and conquer will solve the same sub problem again and agaain \textbf{(Recursion)}
              \item Dynamic programming will solve each sub-problem once, and remembers the answer \textbf{(Memorize)}
          \end{itemize}
    \item Trades space (to save sub-problem solutions) to save time
    \item Usually used to solve \textbf{`optimization'} problems (similar to greedy)
\end{itemize}

\subsection*{Coin Change Problem}

\begin{equation*}
    D_n = D_{n-1} + D_n{n-3} + D_{n-4}
\end{equation*}

\subsubsection*{Base Cases}
\begin{align*}
    D_0 & = 1                                             \\
    D_1 & = 1 \Rightarrow `1'                             \\
    D_2 & = 1 \Rightarrow `1 + 1'                         \\
    D_3 & = 2 \Rightarrow `1 + 1 + 1' \text{ or } `1 + 2' \\
\end{align*}

\subsection*{Rod Cutting}
Give a rod of length $n$ with \(n - 1\) cutting points,
as well as revenue for each length:

\begin{tabular}{c|c|c|c|c|c|c|c|c|c|c}
    Length $i$  & 1 & 2 & 3 & 4 & 5  & 6  & 7  & 8  & 9  & 10 \\
    \hline
    Price $p_i$ & 1 & 5 & 8 & 9 & 10 & 17 & 17 & 20 & 24 & 30
\end{tabular}

\emph{Find the best cutting for the rod that maximizes the revenue.}

For a rod of length 4, what's the best cut?

\begin{itemize}
    \item 1, 3
    \item 2, 2
\end{itemize}

How many possible cuttings for rod with length $n$?

\begin{itemize}
    \item Exponential \(\rightarrow 2^{n-1} = 2^3 = 8\)
\end{itemize}

Each cutting point is a random variable (0 or 1)
\begin{itemize}
    \item We have \(n-1\) cutting points
    \item The total possibilities is \(2^{n-1}\)
\end{itemize}

\subsubsection*{Naive Approach}
\begin{itemize}
    \item Try all possibilities and select the max
    \item Best choice: two pieces each of size 2 \(\Rightarrow revenue = 5 + 5 = 10\)
\end{itemize}

\subsection*{Optimizaiton Problem}
\begin{itemize}
    \item Rod cutting is an optimizaiton problem (maximize profit).
    \item Has optimal substructure property:
    \item You must have the optimal cut for each sub-problem to get the global optimal.
\end{itemize}

\begin{itemize}
    \item Has recursive exponential solution
    \item Has polynomial dynamic programming solution
\end{itemize}

Define the cost (revenue): \(r_i\) is the max revneue for a rod of length \(i\).

\begin{tabular}{ccc}
    $i$ & \(r_i\) & optimal \\
    \hline
    1   & 1       & 1       \\
    2   & 5       & 2       \\
    3   & 8       & 3       \\
    4   & 10      & 2, 2    \\
    5   & 13      & 2, 3    \\
    6   & 17      & 6       \\
    7   & 18      & 1, 6    \\
    8   & 22      & 2, 6    \\
    9   & 25      & 3, 6    \\
    10  & 30      & 10      \\
\end{tabular}

\begin{equation*}
    r_n = \text{max}(p_n, r_1 + r_{n-1}, r_2 + r_{n-2}, \ldots, r_{n-1} + r_1)
\end{equation*}

\subsection*{Basic Approach: Recursive}

First, cut a piece off the left of the rod, and sell it.
Then, find the optimal way to cut the remainder of the rod.

\begin{equation*}
    r_n = \text{max}_{1 \leq i \leq n} (p_i + r_{n-i})
\end{equation*}

Leaves only one subproblem to solve rather than two subproblems.

\begin{verbatim}
Cut-Rod(p, n)
if n == 0
    return 0
q = -infinity
for i = 1 to n
    q = max(q, p[i] + Cut-Rod(p, n-i))
return q
\end{verbatim}

Similar disadvantages to fibonacci recursive solution.

\begin{equation*}
    T(n) = \begin{cases}
        1 \quad                         & \text{if } n = 0 \\
        1 + \sum_{j=0}^{n-1} T(j) \quad & \text{if } n > 0
    \end{cases}
\end{equation*}

\subsubsection*{Memorizing}
\begin{verbatim}
Memoized-Cut-Rod(p, n)
let r[0..n] be a new array
for i = 0 to n
    r[i] = -infinity
return Memoized-Cut-Rod-Aux(p, n, r)
\end{verbatim}

\begin{verbatim}
Memoized-Cut-Rod-Aux(p, n, r)
if r[n] >= 0
    return r[n]
if n == 0
    q = 0
else
    q = -infinity
    for i = 1 to n
        q = max(q, p[i] + Memoized-Cut-Rod-Aux(p, n-i, r))
r[n] = q
return q
\end{verbatim}

\begin{itemize}
    \item Solves each subproblem only once
    \item Solves subproblems for sizes \(0, 1, 2, \ldots, n\)
    \item To solve subproblem of size $n$, the for loop iterates $n$ times
    \item Overall recursive calls, the total number of iterations \(= 1 + 2 + \ldots\)
    \item \(\Theta(n^2)\) time
\end{itemize}

\begin{verbatim}
Bottom-Up-Cut-Rod(p, n)
let r[0..n] be a new array
r[0] = 0
for j = 1 to n
    q = -infinity
    for i = 1 to j
        q = max(q, p[i] + r[j-i])
    r[j] = q
return r[n]
\end{verbatim}

\begin{itemize}
    \item Nested loops, \(1 + 2 + 3 + \ldots + n\)
    \item \(\Theta(n^2)\) time
\end{itemize}

Bottom up is \emph{probably} easier to code.

These algorithms tell you the optimal revenue, but not \emph{how to get it}.
\end{document}