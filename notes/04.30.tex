\documentclass{article}

\usepackage{amsmath, amssymb}

\begin{document}
\section*{DFS: Specification}

\begin{description}
    \item[Input]  Directed or undirected graph \(G = (V, E)\)
    \item[Output] For each \(v\), keep two timestamps and the predecessor:
        \begin{itemize}
            \item \(v.d\) = discovery time
            \item \(v.f\) = finishing time
            \item \(v.\pi\) = predecessor of \(v\) in the depth-first `forest'
        \end{itemize}
\end{description}

BFS gives us a shortest-path.
DFS can be used in additional algorithms.


\subsection*{Depth-First Forest and Breadth-First Tree}
\begin{itemize}
    \item \(DFS(G)\) is usually for finding relationship among verticies (timestamps), not the relationship w.r.t a particlar source \\
          Use \(DFS(G, s)\) as a subroutine
    \item \(BFS(G, s)\) is usually for finding shortest path distances from a given source.
\end{itemize}

\subsection*{Classification of Edges}
\begin{description}
    \item[Tree Edge] In the depth-first forest. Found by exploring \((u, v)\)
    \item[Back Edge] \((u, v)\), where \(u\) is a descendant of \(v\) (in dft)
    \item[Forward Edge] \((u, v)\) where \(v\) is a descendant of \(u\) but not a tree edge
    \item[Cross Edge] any other edge
\end{description}
\end{document}