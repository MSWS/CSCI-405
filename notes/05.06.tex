\documentclass{article}

\usepackage{geometry}[margin=1in]
\usepackage{amsmath, amssymb}

\begin{document}
\section*{Minimum Spanning Tree}
Spanning tree of a connected undirected graph \(G\) = a subgraph that is a tree
and connects all verticies.

There can be many spanning trees of a graph.

BFS and DFS both generate spanning trees BFS is typically `short and bushy' DFS
is typically `long and stringy'

A \emph{minimum spanning tree} is a spanning tree of a graph with the smallest
weight.

\begin{equation*}
    Weight = \sum_{edges} weight(edge)
\end{equation*}

\begin{itemize}
    \item A town has a set of houses and a set of roads
    \item A road connects 2 and only 2 houses
    \item A road connecting houses u and v has a repair cost \(w(u, v)\)
\end{itemize}

Goal: Repair enough (and no more) roads such that
\begin{enumerate}
    \item Everyone stays connected: can reach every hoes from all other houses
    \item Total repair cost is minimized
\end{enumerate}

For an unweighted graph, any spanning tree is a minimum spanning tree.

Finding an MST is an optimization problem Two greedy algorithms:

\begin{description}
    \item[Kruskal's]  consider edges in ascending order, at each step select the
        next edge as long as it does not create cycle
    \item[Prim's] start with any vertex S and greedily grow a tree from S. At
        each step, add the edge of th eleast weight to connect an isolated vertex.
\end{description}

\begin{verbatim}
Kruskal's Algorithm
start with T = V (no edges)
for each edge in increasing order by weight
    if adding edge does not create a cycle
        add edge to T
\end{verbatim}

\begin{verbatim}
MST-KRUSKAL(G, w) //  w = weights
A = {}
for each vertex v in G.V
    MAKE-SET(v)
sort the edges of G.E into nondecreasing order by weight w
for each edge (u, v) in G.E, taken in nondecreasing order by weight
    if FIND-SET(u) != FIND-SET(v)
        A = A U {(u, v)}
        UNION(u, v)
return A
\end{verbatim}

\subsection*{Runtime Analysis}
Kruskal Running time
\begin{align*}
    \text{Sorting }                & = O(E \log E) = O(E \log v^2) = O(2E \log V) = O(E \log V)     \\
    \text{Disjoint-set operations} & = O(m \alpha(n)) = O((2V + 2E - 1) \alpha(V)) = O(E \alpha(V)) \\
    O(E \log V) + O(E \alpha(V))   & = O(E \log V)
\end{align*}

MSTs are unique only if all edge weights are distinct.
\end{document}
