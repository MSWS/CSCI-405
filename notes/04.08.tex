\documentclass{article}

\usepackage{amsmath, amssymb}

\newcommand{\runtimeselect}{\sum_{\text{Phase}_j}X_j \cdot c \cdot \left( \frac{3}{4} \right) ^ j n}

\begin{document}
\section*{Review}
$I$-th order statistic $\rightarrow$ selection problem.

Divice and Conquer (recursion)

\begin{description}
    \item[Worst Case:]  If we always partition around the largest/smallest remaining element.
        \begin{equation*}
            T(n) = \underbrace{O(1)}_{\text{Choose the pivot}} + \underbrace{\Theta(n)}_{\text{Partition}} + T(n-1)
        \end{equation*}
\end{description}

If RSelect randomly chooses a `good pivot' giving at least a 25-75 split, it can be good enough for \(O(n)\) runtime.

\textbf{What is 25-75?} A split that separates the array into two parts, one of which is at least 25\% of the size of the original array.

\subsection*{Phases}
\begin{equation*}
    \text{RSelect is in Phase } j \text{ if current array size is between } \left( \frac{3}{4} \right) ^ {j+1} n \text{ and } \left( \frac{3}{4} \right) ^ j n
\end{equation*}

Note that we will be starting from Phase 0, as we need to initially look at the entire array.
Thus:
\begin{align*}
    \left( \frac{3}{4} \right) ^ {0+1} n = \frac{3}{4}n \\
    \left( \frac{3}{4} \right) ^ {0} n = n
\end{align*}

Vs starting from Phase 1:
\begin{align*}
    \left( \frac{3}{4} \right) ^ {1+1} n = \frac{9}{16}n \\
    \left( \frac{3}{4} \right) ^ {1} n = \frac{3}{4}n
\end{align*}

Logically, the \# of recursive calls in Phase 0 is 2.

To reiterate,

\begin{equation*}
    \text{Running time of RSelect} \leq \runtimeselect
\end{equation*}

\begin{enumerate}
    \item Phase 0: array size between \(\rightarrow n \text{ and } \frac{3}{4}n\)
\end{enumerate}

\textbf{Why 25-75?} Why not 20-80? \\
We could use any range, but using a 25-75 gives us an easy way to compare to flipping coins (50\% chance).

\subsection*{Bernoulli Trial}
An experiment with only two outcomes: success with probability $p$, and failrue, with probability \(q = 1 - p\).

\begin{equation*}
    E[N] = \frac{1}{p} = 2\ (\text{Recall } E[X_j] \leq E[N])
\end{equation*}

Examples
\begin{itemize}
    \item Pulling a specific card out of a deck of cards
    \item Flipping a coin
    \item Winning Rock Paper Scissors
\end{itemize}

\begin{align*}
    \text{E[\text{Running Select}]} & \leq E \left[ \runtimeselect \right]                          \\
                                    & = cn \sum_{\text{Phase} j} \left( \frac{3}{4}\right)^j E[X_j] \\
                                    & = 2n \sum_{\text{Phase}_j} \left( \frac{3}{4}\right)^j        \\
                                    & \leq 8cn = O(n)
\end{align*}

\subsection*{Guaranteeing a 25-75 Split}
\begin{itemize}
    \item What is a good pivot? A balanced split
    \item `Best' pivot? Median
    \item We need a method to deterministally find a good approxmiation of the median.
\end{itemize}

% We're using a deterministic approach, here's how we find the median:
\textbf{Key Idea:} Median of medians.

\subsection*{Deterministic Choose Pivot}
\begin{enumerate}
    \item Divide elements into groups of five; last group may have fewer than five elements.
    \item Sort each group (eg. using MergeSort)
    \item Copy \(n/5\) medians into new array $C$.
    \item Make recursive call to get the median of $C$.
    \item Use this median as the pivot.
    \item If the pivot is not the order statistic that is searched for, recurse on the sub-array that contains it.
\end{enumerate}

\begin{verbatim}
DSelect(array A, length n, order statistic i)
    Break A into groups of 5, sort each gruop
    C = the n/5 `middle elements'
    p = DSelect(C, n/5, n/10)
    Partition A around p
    if j = 1 return p
    if j < i return DSelect(1st part of A, j-1, i)
    return DSelect(2nd part of A, n-j, i-j)
\end{verbatim}

\textbf{What's the running time of step 1 of this algorithm?}
\begin{enumerate}
    \item \(\Theta(1)\)
    \item \(\Theta(n \log n)\)
    \item \(\star\ \Theta(n)\)
    \item  \(\Theta(\log n^n)\)
\end{enumerate}

Lemma: For every input array of $n$ numbers, Merge Sort produces a sorted output array and at most \(6n \log_2 n + 6n\) operations.
\begin{equation*}
    6n \log_2 n + 6n = 30 \log_2 5 + 30 \leq 120
\end{equation*}
\end{document}