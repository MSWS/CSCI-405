\documentclass{article}

\usepackage{amsmath, amssymb}

\begin{document}
Shortest path has optimal substructure.

\begin{equation*}
    \delta(a, b) = \min(\delta(a, c) + w(c, b), \delta(a, d) + w(d, b))
\end{equation*}

Given a weighted, directed graph G = (V, E) and a source vertex s in V, find the
min cost path from so to every vertex in V.

\begin{itemize}
    \item Bellman-Ford
    \begin{itemize}
        \item DP
        \item General case, edge weights may be negative
    \end{itemize}
    \item Djiikstra
    \begin{itemize}
        \item Greedy
        \item Edge weights must be non-negative
    \end{itemize}
\end{itemize}

Relaxation: Given a vertex v, a vertex u, and an edge (u, v), we can relax v by

\begin{verbatim}
Relax(u, v)
    if v.d > u.d + w(u, v)
        v.d = u.d + w(u, v)
        v.pi = u
\end{verbatim}

Proof by induction on relaxing the ith edge \((v_{i-1}, v_i)\) on p

Let \(w_I = \sum_{1}^{i} w(v_{i=1}, v_i)\). \(W_i\) is the shortest path weights
\(\delta(s, v_i)\) because of optimal substructure


\end{document}