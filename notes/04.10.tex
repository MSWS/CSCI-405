\documentclass{article}

\usepackage{amsmath, amssymb}

\newenvironment{rcases}
  {\left.\begin{aligned}}
  {\end{aligned}\right\rbrace}

\begin{document}
\section*{Review}
\subsection*{The DSelect Algorithm}

Elements that are $>$ or \(< m\)
\begin{equation*}
    3 \left( \lceil \frac{1}{2} \lceil \frac{n}{5} \rceil \rceil - 2\right) \geq \frac{3n}{10} - 6
\end{equation*}

Thus, the \emph{upper bound} of the number of elements we put into the recursive call:
\begin{equation*}
    n - \left(\frac{3n}{10} - 6\right) = \frac{7n}{10} + 6
\end{equation*}

\subsection*{Recurrence}
With this in mind, our recursion complexity looks along the lines of
\begin{equation*}
    T(N) =
    \begin{cases}
        1, N = 1 \\
        T(n) \leq cn + T \left( \frac{n}{5}\right) + T \left( \frac{7n}{10}\right)
    \end{cases}
\end{equation*}

\subsection*{Hope and Check}
Hope: There is some constant a [independent of a] such that \(T(n) \leq an, \forall n \geq 1\)
If true, \(T(n) = O(n)\)

\subsection*{Analysis}
\begin{description}
    \item[Claim:]  Let \(a = 10c\), then \(T(n) \leq an, \forall n \geq 1\)
    \item[Proof:] By induction on n.
    \item[Base Case:] \(T(1) = 1 \leq a \times 1 (n = 1, \text{, since } a \geq 1)\)
    \item[Induction Step:] \(n > 1\)
\end{description}

\begin{equation*}
    \text{Induction Hypotheseis: } T(k) \leq a\times k, \text{ for } k < n
\end{equation*}

\begin{align*}
    \text{We have } T(n) & \leq cn + T(n/5) + T(\frac{7n}{10})               \\
                         & \leq cn + a(n/5) + a \left( \frac{7n}{10} \right) \\
                         & = n(c + \frac{9a}{10}) = an
\end{align*}

\section*{Dynamic Programming}
Dynamic programming is a method for designing efficient algorithms for
recursively solvable problems with the following properties:

\begin{enumerate}
    \item Optimal Substructure: An optimal solution to an instance contains an
          optimal solution to its sub-instances
    \item Overlapping Subproblems: The number of subproblems is small so during
          the recursion same instances are referred to over and over again.
\end{enumerate}

\emph{Basically fancy recursion}

Four steps in solving a problem using dynamic programming:
\begin{enumerate}
    \item Characterize the structure
    \item Recursively define the value of a solution
    \item
\end{enumerate}

A more refined / intuitive list: \footnote{Recursion + memorization}
\begin{enumerate}
    \item Define subproblems
    \item Write down the recurrence that relates the subproblems
    \item Recognize and solve the best cases
\end{enumerate}

\subsection*{Fibonacci}
Sequence: \(1, 1, 2, 3, 5, 8, 13, \ldots\)

\subsubsection*{Naive Recursion Approach}
\begin{verbatim}
def fib(n):
    if n <= 2:
        return 1
    return fib(n - 1) + fib(n - 2)
\end{verbatim}

\subsubsection*{Runtime} \(O(2^n)\)
We calculate the same value multiple times.

\begin{equation*}
    T(n) = T(n - 1) + T(n - 2) + 2
\end{equation*}

\subsubsection*{Memoized DP Approach}
\begin{verbatim}
memo = {}
def fib(n):
    if n in memo: return memo[n]
    if n <= 2:
        f = 1
    else
        f = fib(n - 1) + fib(n - 2)
    memo[n] = f
    return f
\end{verbatim}

\subsubsection*{Bottom-Up DP Approach}
\begin{verbatim}
fib = {}
for k in range(n):
    if k <= 2:
        f = 1
    else:
        f = fib[n - 1] + fib[n-2]
    fib[k] = f
return fib[n]
\end{verbatim}

\subsection*{Coin Change}
Given $n$, find the number of different ways to write $n$ as the sum of \(1, 3,
4\)

Example: \(n = 5\)
\begin{align*}
    5 & = 1+ 1 + 1 + 1 + 1 \\
      & = 1 + 1 + 3        \\
      & = 1 + 3 + 1        \\
      & = 3 + 1 + 1        \\
      & = 1 + 4            \\
      & = 4 + 1
\end{align*}
\begin{equation*}
    \begin{aligned}
        \begin{rcases}
            5 & = 1 + 1 + 1 + 1+ 1 \\
              & = 3 + 1 + 1        \\
              & = 1 + 3 + 1        \\
              & = 4 + 1            \\
        \end{rcases}  D_4 \\
        = 1 + 1 + 3
    \end{aligned}
\end{equation*}


\subsubsection*{Define Subproblems}
\begin{equation*}
    \text{Let} D_n \text{ be the number of ways to write} n \text{ as the sum of } 1, 3, 4
\end{equation*}

\subsubsection*{Find the recurrence}
\begin{enumerate}
    \item Consider one possible solution \(n = x_1 + x_2 + \cdot + x_m\)
    \item If \(x_m = 1\), the rest of the terms must sum to \(n - 1\)
    \item Thus, the number of sums that end with \(x_m = 1\) is equal to \(D_{n-1}\)
    \item Take other cases into account \((x_m = 3, x_m = 4)\)
\end{enumerate}

\begin{equation*}
    D_n = D_{n-1} + D_{n-3} + D_{n-4}
\end{equation*}

\subsubsection*{Base Cases}
\begin{align*}
    D_0 & = 1 & D_1 & = 1
\end{align*}
\end{document}
