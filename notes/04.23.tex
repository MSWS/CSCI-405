\documentclass{article}

\usepackage{amsmath, amssymb}

\begin{document}
\section*{Properties of Greedy Algorithm}
Problems that can be solved by greedy algorithms have two main properties:
\begin{description}
    \item[Optimal Substructure] The optimal solution to the problem contains within it optimal solutions to subproblems.
    \item[Greedy Choice Property] A global optimal solution can be arrived at by selecting a local optimal solution.
\end{description}

Difference from DP: No overalapping subproblems.

\section*{Activity Selection Problem}

\begin{description}
    \item[Input] Set \(S\) of \(n\) activities \(a_1, a_2, \ldots, a_n\).
        \begin{itemize}
            \item \(s_i\) = start time of activity \(i\)
            \item \(f_i\) = finish time of activity \(i\)
        \end{itemize}
    \item[Output] Subset \(A\) of maximum number of compatible activities.
\end{description}
Two activities are compatible if they do not overlap.

\begin{tabular}{c|ccccccccc}
    i       & 1 & 2 & 3 & 4 & 5 & 6  & 7  & 8  & 9  \\
    \hline
    \(s_i\) & 1 & 2 & 4 & 1 & 5 & 8  & 9  & 11 & 13 \\
    \(f_i\) & 3 & 5 & 7 & 8 & 9 & 10 & 11 & 14 & 16 \\
\end{tabular}

Assume activites are sorted by finishing times.
\begin{equation*}
    f_1 \leq f_2 \leq \ldots \leq f_n
\end{equation*}

Suppose an optimal solution includes activity \(a_k\).

\begin{itemize}
    \item This generates two subproblems
          \begin{itemize}
              \item Selecting from \(a_1, \ldots, a_{k-1}\) activites compatible with one another, and that finish before \(a_k\) starts (compatible with \(a_k\))
              \item Selecting from \(a_{k+1}, \ldots, a_n\) activites compatible with one another, and that start after \(a_k\) finishes.
          \end{itemize}
    \item The solutions to the two subproblems must be optimal.
          \begin{itemize}
              \item Prove using the cut-and-paste approach.
          \end{itemize}
\end{itemize}

\subsection*{Optimal Substructure}
\begin{equation*}
    S_{ij} = \text{ subset of activites in } S \text{ that start after } a_i \text{ finishes and finish before } a_j \text{ starts}
\end{equation*}

\begin{align*}
    \text{Let } A_{ik} & = A_{ij} \cap S_{ik} \text{ and } A_{kj} = A_{ij} \cap S_{kj} \\
    A_{ij}             & = A_{ik} \cup \{a_k\} \cup A_{kj}                             \\
    \vert A_{ij} \vert & = \vert A_{ik} \vert + \vert A_{kj} \vert + 1
\end{align*}

\begin{equation*}
    c[i, j] = \text{ size of max-size subset of mutually compatible activites in } S_{ij}
\end{equation*}

\begin{equation*}
    \parbox{5em}{Recursive\\Solution} c[i, j] = \begin{cases}
        0                                                        & \text{if } S_{ij} = \emptyset    \\
        % \max\{c[i, k] + c[k, j] + 1\}                            & \text{if } S_{ij} \neq \emptyset \\
        \underset{a_k \in S_{ij}}{\max}\{c[i, k] + c[k, j] + 1\} & \text{if } S_{ij} \neq \emptyset
    \end{cases}
\end{equation*}

After you choose \(a_k\), the subproblem focuses on the set of
\begin{equation*}
    S_k = \{ a_i \in S : s_i \geq f_k\}
\end{equation*}

\textbf{Theorem}
If \(S_k\) is nonempty and \(a_m\) has the earliest finish time in \(S_k\), then
\(a_m\) is included in some optimal solution.

\textbf{Proof} Let \(A_k\) be an optimal solution to \(S_k\), and let \(a_j\)
have the earliest finish time of any activity in \(A_k\). If \(a_j = a_m\),
done. Otherwise, let \(A_k' = A_k - \{ a_j \} \cup \{ a_m \}\) be \(A_k\) but
with \(a_m\) substituted for \(a_j\).

\begin{verbatim}
Recursive-Activity-Selector(s, f, k, n)
    m = k + 1
    while m <= n and s[m] < f[k]
        m = m + 1
    if m <= n
        return {m} U Recursive-Activity-Selector(s, f, m, n)
    else
        return {}
\end{verbatim}
\end{document}