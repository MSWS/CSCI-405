\documentclass{article}

\usepackage{amsmath, amssymb}

\begin{document}
Working mod \(q = 11\), how many spurious hits does the Rabin-Karp matcher
encounter in \(S = 3141592653589793\) when looking for \(P = 26\)

3

\section*{Knuth-Morris-Pratt}
\begin{itemize}
    \item Best known for linear time for exact matching
    \item Compares from L-R but shifts more than one position
    \item Preprocessing appraoch of pattern to avoid trivial comparisons
    \item Conceived by Donald Knuth and Vaughan Pratt
    \item Guaranteed worst-case efficiency \(\Theta((n - m + 1) m)\)
    \item Preprocessing time is \(O(m)\)
    \item Searching time is \(O(n)\)
\end{itemize}

Two main aspects:
\begin{description}
    \item[Pre-processing]  Involves parsing through the pattern alone \((O(m))\)
        time and space- an array of prefix-suffix match is created
    \item[Searching] Involves parsing though the string using the pre-processed
        array and the patern array \((O(n))\) time
\end{description}

\(T= AAABAAAB, P=AAAA\)
\begin{itemize}
    \item Compute in advance how far to jump in P (Pre-processing) when a match
          fails
    \item Retain information from prior attempts
    \item Never decrement \(i\), ever.
\end{itemize}

\begin{quote}
    The length of the longest proper prefix in the (sub)pattern that matches a
    proper suffix in the same (sub)pattern
\end{quote}
\begin{align*}
    LPS[0,?,?,?] &                   \\
    LPS[0,1,?,?] & P = \mathbf{AA}AA \\
    LPS[0,1,2,?] & P = \mathbf{AAA}A \\
    LPS[0,1,2,3] & P = \mathbf{AAAA} \\
\end{align*}

\(P = AAABAAAA\)
\begin{align*}
    LPS[0,?,?,?,?,?,?,?] &                       \\
    LPS[0,1,?,?,?,?,?,?] & P = \mathbf{A}AABAAAA \\
    LPS[0,1,2,?,?,?,?,?] & P = \mathbf{AA}ABAAAA \\
    LPS[0,1,2,3,?,?,?,?] & P = \mathbf{AAA}BAAAA \\
    LPS[0,1,2,3,3,?,?,?] & P = \mathbf{AAA}BAAAA \\
    LPS[0,1,2,3,3,3,?,?] & P = \mathbf{AAA}BAAAA \\
    LPS[0,1,2,3,3,3,3,?] & P = \mathbf{AAA}BAAAA \\
    LPS[0,1,2,3,3,3,3,3] & P = \mathbf{AAA}BAAAA \\
\end{align*}

How do we get \(LPS[7]\) from \(LPS[6]\)?
\begin{equation*}
    LPS[6] = 2: P = \mathbf{AA}AB\mathbf{AA}AA
\end{equation*}
Check if \(P[7] = P[3]\): Yes \(LPS[7] = LPS[6] + 1\)

\(P=ABACABAB\)
\(LPS=[0,0,1,0,1,2,3,2]\)
\(LPS[7] = 3: P = \mathbf{ABA}C\mathbf{ABA}B\)
Check if \(P[8](B) = P[4](C)\) No! (Not able to create longer P/S)
We can make use of the info \([0, 0, \mathbf{1}, 0, 1, 2, 3, 2]\)
Meanining \(P[1] = P[7] (P[1] = P[3] = P[5] = P[7])\)
\end{document}