\documentclass{article}

\usepackage{amsmath, amssymb}
\usepackage{enumitem}

\begin{document}
\section*{Review}
\texttt{DSelect} $\rightarrow$ Deterministic Select

Goals of DSelect:
\begin{itemize}
    \item Try and get the median of the medians
\end{itemize}

\subsection*{Runtimes}
\begin{tabular}{cc|c}
            & DSelect  & RSelect  \\
    \hline
    Best    &          &          \\
    Worst   & \(O(n)\) &          \\
    Average & \(O(n)\) & \(O(n)\) \\
\end{tabular}

How many recursive calls does DSelect make? \textbf{2} \\
One for computing the median of C, and one for the partitioning step.

\subsubsection*{First Step}
\begin{align*}
     & \leq \frac{n}{5} 120 = 24n \\
     & \Rightarrow O(n)
\end{align*}

\subsubsection*{Generalized}
\begin{equation*}
    T(n) = 3 \Theta(n) + T \left( \frac{n}{5} \right)
\end{equation*}

\subsection*{Rough Recurrence}
Let \(T(n) = \) max runtime of DSelect on an input array of length $n$.

\begin{align*}
    \exists c \geq \text{ such that:} \\
    T(1) = 1                          \\
    T(n) \leq \underbrace{cn}_{\text{Sorting + Partition}} +  \underbrace{T\left( \frac{n}{5}\right)}_{\text{Recurse to get good pivot}} + \underbrace{T(?)}_{\text{Recurse in Line 6/7}}
\end{align*}

Goal: for every input array of length $n$, the runtime of DSelect is at most
$cn$. \\
Note: Not as good as RSelect in practice.
Reasons:
\begin{enumerate}
    \item Constant hidden in the Big-O
    \item Extra memory to store the C-array
\end{enumerate}

Why do we use groups of 5? \\
Using an even number would make it harder to find the median of the medians.

Using 5, we know that the median of the medians is bigger than 3 out of 5 (60\%) of th eelements in \(\approx 50\%\) of the groups.

Bigger than \(30\%\) of all elements.

\subsection*{Sorted}
Number of groups for $n$ elements: \(\lceil \frac{n}{5} \rceil = 6 \) \\
\(\frac{1}{2} \lceil \frac{n}{5} \rceil + 1\)

At least half of the groups have a median that is bigger than the median of the medians.

2/5

\begin{equation*}
    % \text{At least }\lceil \frac{2m}{5} + 1 \rceil \text{ elements are } > m
    \text{At least }3 \left(\lceil \frac{1}{2} \lceil \frac{n}{5} \rceil \rceil - 2\right) \text{ elements are } > m
\end{equation*}
\end{document}