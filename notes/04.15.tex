\documentclass{article}
\usepackage{amsmath, amssymb}
\usepackage{forest}
\usepackage{tikz}
\usepackage{graphicx}
\usepackage[utf8]{inputenc}
\usepackage{pmboxdraw}

\newtheorem{theorem}{Theorem}

\begin{document}
\section*{Rod Cutting}
\subsection*{Reconstructing the optimal solution}

\begin{verbatim}
    Extended-Bottom-Up-Cut-Rod(p, n)
    let r[0..n] and s[0..n] be new arrays
    r[0] = 0
    for j = 1 to n
        q = -infinity
        for i = 1 to j
            if q < p[i] + r[j - i]
                q = p[i] + r[j - i]
                s[j] = i
        r[j] = q
    return r and s
\end{verbatim}

\texttt{s[j] = 1} tracks the best cut location so far for length \(j\).

\begin{tabular}{ccccccccccr}
    i        & 0 & 1 & 2 & 3 & 4  & 5  & 6  & 7  & 8                              \\
    \hline
    \(r[i]\) & 0 & 1 & 5 & 8 & 10 & 13 & 17 & 18 & 22 & Max Revenue               \\
    \(s[i]\) & 0 & 1 & 2 & 3 & 2  & 2  & 6  & 1  & 2  & \(1^{{st}}\) cut position \\
\end{tabular}

\subsection*{Optimal Substructre}
\begin{itemize}
    \item Create an optimal solution to a problem using optimal solutions to subproblems.
    \item We can't use DP if the optimal solution to a problem might not require subproblem solutions to be optimal \\
          \(\Rightarrow\) This often happens when the subproblemsn are \textbf{not independent} of each other.
\end{itemize}

\subsection*{Overlapping Subproblems}
\begin{itemize}
    \item For DP to be useful, the recursive algorihtm should require us to compute optimal solutions to the \textbf{same subproblems} over and over again.
    \item In total, there should be a small number of distinct subproblems (i.e. polynomial in the input size).
\end{itemize}

DP can both be \textbf{top-down} or \textbf{bottom-up}.

\section*{2-Dimensional Example: Longest Common Subsequence}
Problem: Give 2 sequences, \(X = <x_1, x_2, \dots, x_m>\) and \(Y = y_1, y_2,
\dots, y_n\), find a common subsequence whose lengith is maximum.
Subsequence need not be consecutive, but \emph{must be in order}.
\begin{tabular}{c|c|c}
    X               & Y             & X \(\cap\) Y \\
    \hline
    springtime      & printing      & printi       \\
    ncaa tournament & north carlina & ncarna       \\
    basketball      & snoeyink      & se           \\
    ABCBDAB         & BDCABA        & BCBA         \\
\end{tabular}

\subsection*{Naive Appraoch}
For every subsequence of $X$, check whether it's a subsequence of $Y$.

Suppose \(m\) is the length of \(X\) and \(n\) is the length of \(Y\). There are \(2^m\) subsequences of \(X\), so the time complexity is \(O(2^m)\).

\begin{equation*}
    \text{Runtime} = \Theta(n\cdot2^m)
\end{equation*}

\subsection*{Optimal substructure in LCS}
\begin{itemize}
    \item The last elements of $X$ and $Y$ are equal. Then they must both be part of the LCS.
    \item So we can chop both elements off the ends of the subsequence (adding
          them to a common subsequence) and find the longest common subsequence of the
          smaller sequences.
\end{itemize}

\begin{align*}
    \text{If } X = ABCBDAB \quad Y = BDCABAB \qquad Z = <z_1, \ldots,  B> \\
    X = ABCBDA \quad Y = BDCABA \qquad Z = <z_1, \ldots, A, B>
\end{align*}
\begin{description}
    \item[Subproblem 1] \(X = ABCBD\) and \(Y = BDCA\)
    \item[Subproblem 2] \(X = ABCB\) and \(Y = BDCAB\)
\end{description}

\begin{theorem}
    Let \(Z = <z_1, \ldots, z_k>\) be any LCS of \(X\) and \(Y\).
    \begin{enumerate}
        \item If \(x_m = y_n\), then \(z_k = x_m = y_n\) and \(Z_{k-1}\)
    \end{enumerate}
\end{theorem}

Define \(c[i, j] = \) length of LCS of \(X_i\) and \(Y_j\). We want \(c[m, n]\).
\begin{equation*}
    c[i, j] = \begin{cases}
        0                          & \text{if } i = 0 \text{ or } j = 0,           \\
        c[i-1, j-1] + 1            & \text{if } i, j > 0 \text{ and } x_i = y_j,   \\
        \max(c[i, j-1], c[i-1, j]) & \text{if } i, j > 0 \text{ and } x_i \neq y_j
    \end{cases}
\end{equation*}

\begin{equation*}
    c[i, j] = \begin{cases}
        0                          & \text{if } i = 0 \text{ or } j = 0,           \\
        c[i-1, j-1] + 1            & \text{if } i, j > 0 \text{ and } x_i = y_j,   \\
        \max(c[i, j-1], c[i-1, j]) & \text{if } i, j > 0 \text{ and } x_i \neq y_j
    \end{cases}
\end{equation*}

% \begin{forest}
%     [ c(springtime, printing)
%         [ c(springtim, printin)
%                 [ c(springti, printi)
%                         [ c(springt, print)
%                                 [ c(spring, prin)
%                                         [ c(sprin, pri)
%                                                 [ c(spri, pr)
%                                                         [ c(spr, p)
%                                                                 [ c(sp, p)
%                                                                         [ c(s, p)
%                                                                                 [ c(, p) ]
%                                                                                     [ c(s, ) ] ]
%                                                                             [ c(sp, ) ] ]
%                                                                     [ c(s, p) ] ]
%                                                             [ c(spr, pr) ] ]
%                                                     [ c(sprin, pri) ] ]
%                                             [ c(spring, prin) ] ]
%                                     [ c(springt, print) ] ]
%                             [ c(springti, printi) ] ]
%                     [ c(springtim, printin) ] ]
%     ]
% \end{forest}

\begin{equation*}
    LCS[i, j] = \begin{cases}
        1 + LCS[i - 1, j-1] & \text{if } x_i = y_j, \\
    \end{cases}
\end{equation*}

\subsection*{Recursive Solution (for getting length)}

\begin{tabular}{c|c|ccccccc}
      & j       & 0       & 1 & 2 & 3 & 4 & 5 & 6 \\
    \hline
    i &         & \(y_j\) & B & D & C & A & B & A \\
    0 & \(x_i\) & 0       & 0 & 0 & 0 & 0 & 0 & 0 \\
    1 & A       & 0       & 0 & 0 & 0 & 1 & 1 & 1 \\
    2 & B       & 0       & 1 & 1 & 1 & 1 & 2 & 2 \\
    3 & C       & 0       & 1 & 1 & 2 & 2 & 2 & 2 \\
    4 & B       & 0       & 1 & 1 & 2 & 2 & 3 & 3 \\
    5 & D       & 0       & 1 & 2 & 2 & 2 & 3 & 3 \\
    6 & A       & 0       & 1 & 2 & 2 & 3 & 3 & 4 \\
    7 & B       & 0       & 1 & 2 & 2 & 3 & 4 & 4 \\
\end{tabular}

\begin{verbatim}
LCS-Length(X, Y)
    m <- length[X]
    n <- length[Y]
    c[0, 0] <- 0
    for i <- 1 to m
        c[i, 0] <- 0
    for j <- 1 to n
        c[0, j] <- 0
    for i <- 1 to m
        for j <- 1 to n
            if x[i] = y[j]
                c[i, j] <- c[i-1, j-1] + 1
            elseif c[i-1, j] >= c[i, j-1]
                c[i, j] <- c[i-1, j]
            else
                c[i, j] <- c[i, j-1]
    return c[m, n]
\end{verbatim}

This returns the length. how do we get the actual LCS?
\begin{verbatim}
LCS-Length(X, Y, m, n)
    let b[1..m, 1..n] and c[0..m, 0..n] be new tables
    for i = 1 to m
        c[i, 0] = 0
    for j = 1 to n
        c[0, j] = 0
    for i = 1 to m
        for j = 1 to n
            if x[i] = y[j]
                c[i, j] = c[i-1, j-1] + 1
                b[i, j] = "north-west arrow"
            elseif c[i-1, j] >= c[i, j-1]
                c[i, j] = c[i-1, j]
                b[i, j] = "↑"
            else
                c[i, j] = c[i, j-1]
                b[i, j] = "←"
    return c and b
\end{verbatim}

\end{document}