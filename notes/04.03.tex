\documentclass{article}

\usepackage{amsmath, amssymb}

\usepackage[utf8]{inputenc}

\begin{document}
\section*{Maximum and Minimum}
The \textbf{selection problem} is the problem of computing the $i$-th order
statistic (i.e. the $i$-th smallest number) of a set $A$.

Special Cases:
\begin{itemize}
    \item Min ($i = 1$)
    \item Max ($i = n$)
    \item Median ($i = n/2$)
\end{itemize}

\subsection*{Order Statistics}
How many comparisons are necessary and sufficient for computing both the min and max?

\begin{description}
    \item[Necessary:] you cannot find it with fewer comparisons
    \item[Sufficient:] no more comparisons are needed
\end{description}

\begin{itemize}
    \item Compute the max: $n - 1$ comparisons
    \item Compute the min: $n - 1$ comparisons
    \item Thus, computing both min \& max requires $2n - 2$ comparisons
\end{itemize}

\textbf{Is there a way to improve this?}

\subsection*{Idea}
Maintain the variables min and max. Process the $n$ numbers in pairs.

Operation:
\begin{itemize}
    \item Form pairs of elements
    \item Compare elements in each pair
\end{itemize}

\subsection*{Initial Values}
\begin{description}
    \item[$n$ is odd] set both min and max to the first elements
    \item[$n$ is even] compare first two elements, assign smallest to min and largest to max
\end{description}


Total \# of comparisons:
\begin{description}
    \item[n is od] $3(n-1)/2$
    \item [n is even] $1 + 3(n-2)/2 = 3n/2 - 2 $
\end{description}

\subsection*{Odd Example}
n = 5(odd), array $A = {2, 7, 1, 3, 14}$

\begin{enumerate}
    \item Set min = max = 2
    \item Compare elements in pairs \begin{itemize}
              \item $1 < 7 \implies $ compare 1 with min and 7 with max
              \item $3 < 5 \implies$ compare 3 with min and 4 with max
          \end{itemize}
          We performed $3(n-1)/2 = 6$ comparisons
\end{enumerate}

\subsection*{Even Example}
n = 6(even), array $A = {2, 5, 3, 7, 1, 4}$
\begin{enumerate}
    \item Compare 2 with 5: 2 < 5
    \item Set min = 2, max = 5
    \item Compare elements in pairs: \begin{itemize}
              \item $3 < 7 \implies$ compare 3 with min and 7 with max
              \item $1 < 4 \implies$ compare 1 with min and 4 with max
          \end{itemize}
          We performed $1 + 3(n-2)/2 = 7$ comparisons
\end{enumerate}

The total \# of comparisons is at most

\begin{equation*}
    3 \lfloor n / 2 \rfloor
\end{equation*}

\subsection*{What about the median?}
\begin{itemize}
    \item By repeatedly applying the algorihtm for finding min value, it will
    take $O(n)$ time to find the $i$-th smallest element.
    \item Therefore, when finding the median, it will take $O(n^2)$ time. This
    is more than what sorting takes.
    \item Can we do better?
\end{itemize}

\subsection*{Reduction to sorting}

\begin{enumerate}
    \item $O(n \log n)$ algorithm:
    \item Step 1: apply merge sort
    \item Return $i$-th element of the array
\end{enumerate}

Definitely better than finding the median using the min algorithm.
\textbf{Key Idea}: We can use reduction to sorting to help us with the selection problem
\end{document}