\documentclass{article}

\usepackage{amsmath, amssymb}
\usepackage{forest}
\usepackage{multicol}

\begin{document}
\section*{Optimal Binary Search Trees}

\begin{multicols}{2}
    \begin{minipage}{\columnwidth}
        \begin{center}
            \begin{forest}
                [
                    \(k_2\)
                    [\(k_1\)]
                        [\(k_5\)
                            [\(k_4\)
                                    [
                                            \(k_3\)
                                        ]
                                        [,phantom]
                                ]
                                [,phantom]
                        ]
                ]
            \end{forest}
        \end{center}
    \end{minipage}
    \columnbreak
    \begin{tabular}{ccc}
        \(i\) & \(\text{depth}_T(k_i)\) & \(\text{depth}_T(k_i)\cdot p_i\) \\
        \hline
        1     & 1                       & 0.25                             \\
        1     & 0                       & 0                                \\
        3     & 2                       & 0.1                              \\
        4     & 1                       & 0.2                              \\
        5     & 2                       & 0.3                              \\
        \hline
              &                         & 1.10
    \end{tabular}
\end{multicols}

\begin{equation*}
    \text{Therefore, } E[\text{search cost}] = 2.10
\end{equation*}

\subsection*{Observerations}
\begin{itemize}
    \item Optimal BST may not have smallest height
    \item Optimal BST may not have highest-probability key at root
\end{itemize}

For each, assign keys and compute expected search cost.
But there are \(\Omega(\frac{4^n}{3^{3/2}})\) different BSTs with \(n\) nodes.

\subsection*{Optimal Substructure}
\begin{itemize}
    \item Any subtree of a BST contains key in a contiguous range \(k_i, \ldots,
          k_j\) for some \(1 \leq i \leq j \leq n\).
    \item If \(T\) is an optimal BST and \(T\) contains subtree \(T'\) with keys
          \(k_i, \ldots, k_j\), then \(T'\) must be an optimal BST for keys \(k_i,
          \ldots, k_j\).
\end{itemize}

To find an optimal BST:
\begin{enumerate}
    \item Examine all candidate roots \(k_r, \text{ for } i \leq r \leq j\)
    \item Determine all optimal BSTs containing \(k_i, \ldots, k_{r-1}\) and
          containing \(k_{r+1}, \ldots, k_j\)
\end{enumerate}

When optimal subtree becomes a subtree of a node:
\begin{enumerate}
    \item Depth of every node in OPT subtree goes up by 1
    \item Expected search cost increases by \[w(i, j) = \sum_{l=i}^{j}p_l\]
\end{enumerate}

If \(k_r\) is the root of an optimal BST for \(k_i, \ldots, k_j\):
\begin{align*}
    e[i, j] & = p_r + \left(e[i, r-1] + w(i, r-1)\right) + (e[r+1, j] + w(r+1, j))       \\
            & = e[i, r-1] + e[r+1, j] + w(i, j)                                          \\
            & \qquad (\text{\footnotesize because \(w(i, j) = w(i, r-1) + p_r + w(r + 1, j)\)})
\end{align*}

But, we don't know \(k_r\). Hence,
\begin{equation*}
    e[i, j] = \begin{cases}
        0                                                                                       & \text{ if } j = i -1 \\
        \underbrace{\text{min}}_{i \leq r \leq j}\left\{e[i, r-1] + e[r+1, j] + w(i, j)\right\} & \text{ if } i \leq j
    \end{cases}
\end{equation*}
\end{document}